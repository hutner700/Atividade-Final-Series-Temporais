% Options for packages loaded elsewhere
\PassOptionsToPackage{unicode}{hyperref}
\PassOptionsToPackage{hyphens}{url}
%
\documentclass[
]{article}
\usepackage{amsmath,amssymb}
\usepackage{iftex}
\ifPDFTeX
  \usepackage[T1]{fontenc}
  \usepackage[utf8]{inputenc}
  \usepackage{textcomp} % provide euro and other symbols
\else % if luatex or xetex
  \usepackage{unicode-math} % this also loads fontspec
  \defaultfontfeatures{Scale=MatchLowercase}
  \defaultfontfeatures[\rmfamily]{Ligatures=TeX,Scale=1}
\fi
\usepackage{lmodern}
\ifPDFTeX\else
  % xetex/luatex font selection
\fi
% Use upquote if available, for straight quotes in verbatim environments
\IfFileExists{upquote.sty}{\usepackage{upquote}}{}
\IfFileExists{microtype.sty}{% use microtype if available
  \usepackage[]{microtype}
  \UseMicrotypeSet[protrusion]{basicmath} % disable protrusion for tt fonts
}{}
\makeatletter
\@ifundefined{KOMAClassName}{% if non-KOMA class
  \IfFileExists{parskip.sty}{%
    \usepackage{parskip}
  }{% else
    \setlength{\parindent}{0pt}
    \setlength{\parskip}{6pt plus 2pt minus 1pt}}
}{% if KOMA class
  \KOMAoptions{parskip=half}}
\makeatother
\usepackage{xcolor}
\usepackage[margin=1in]{geometry}
\usepackage{color}
\usepackage{fancyvrb}
\newcommand{\VerbBar}{|}
\newcommand{\VERB}{\Verb[commandchars=\\\{\}]}
\DefineVerbatimEnvironment{Highlighting}{Verbatim}{commandchars=\\\{\}}
% Add ',fontsize=\small' for more characters per line
\usepackage{framed}
\definecolor{shadecolor}{RGB}{248,248,248}
\newenvironment{Shaded}{\begin{snugshade}}{\end{snugshade}}
\newcommand{\AlertTok}[1]{\textcolor[rgb]{0.94,0.16,0.16}{#1}}
\newcommand{\AnnotationTok}[1]{\textcolor[rgb]{0.56,0.35,0.01}{\textbf{\textit{#1}}}}
\newcommand{\AttributeTok}[1]{\textcolor[rgb]{0.13,0.29,0.53}{#1}}
\newcommand{\BaseNTok}[1]{\textcolor[rgb]{0.00,0.00,0.81}{#1}}
\newcommand{\BuiltInTok}[1]{#1}
\newcommand{\CharTok}[1]{\textcolor[rgb]{0.31,0.60,0.02}{#1}}
\newcommand{\CommentTok}[1]{\textcolor[rgb]{0.56,0.35,0.01}{\textit{#1}}}
\newcommand{\CommentVarTok}[1]{\textcolor[rgb]{0.56,0.35,0.01}{\textbf{\textit{#1}}}}
\newcommand{\ConstantTok}[1]{\textcolor[rgb]{0.56,0.35,0.01}{#1}}
\newcommand{\ControlFlowTok}[1]{\textcolor[rgb]{0.13,0.29,0.53}{\textbf{#1}}}
\newcommand{\DataTypeTok}[1]{\textcolor[rgb]{0.13,0.29,0.53}{#1}}
\newcommand{\DecValTok}[1]{\textcolor[rgb]{0.00,0.00,0.81}{#1}}
\newcommand{\DocumentationTok}[1]{\textcolor[rgb]{0.56,0.35,0.01}{\textbf{\textit{#1}}}}
\newcommand{\ErrorTok}[1]{\textcolor[rgb]{0.64,0.00,0.00}{\textbf{#1}}}
\newcommand{\ExtensionTok}[1]{#1}
\newcommand{\FloatTok}[1]{\textcolor[rgb]{0.00,0.00,0.81}{#1}}
\newcommand{\FunctionTok}[1]{\textcolor[rgb]{0.13,0.29,0.53}{\textbf{#1}}}
\newcommand{\ImportTok}[1]{#1}
\newcommand{\InformationTok}[1]{\textcolor[rgb]{0.56,0.35,0.01}{\textbf{\textit{#1}}}}
\newcommand{\KeywordTok}[1]{\textcolor[rgb]{0.13,0.29,0.53}{\textbf{#1}}}
\newcommand{\NormalTok}[1]{#1}
\newcommand{\OperatorTok}[1]{\textcolor[rgb]{0.81,0.36,0.00}{\textbf{#1}}}
\newcommand{\OtherTok}[1]{\textcolor[rgb]{0.56,0.35,0.01}{#1}}
\newcommand{\PreprocessorTok}[1]{\textcolor[rgb]{0.56,0.35,0.01}{\textit{#1}}}
\newcommand{\RegionMarkerTok}[1]{#1}
\newcommand{\SpecialCharTok}[1]{\textcolor[rgb]{0.81,0.36,0.00}{\textbf{#1}}}
\newcommand{\SpecialStringTok}[1]{\textcolor[rgb]{0.31,0.60,0.02}{#1}}
\newcommand{\StringTok}[1]{\textcolor[rgb]{0.31,0.60,0.02}{#1}}
\newcommand{\VariableTok}[1]{\textcolor[rgb]{0.00,0.00,0.00}{#1}}
\newcommand{\VerbatimStringTok}[1]{\textcolor[rgb]{0.31,0.60,0.02}{#1}}
\newcommand{\WarningTok}[1]{\textcolor[rgb]{0.56,0.35,0.01}{\textbf{\textit{#1}}}}
\usepackage{graphicx}
\makeatletter
\def\maxwidth{\ifdim\Gin@nat@width>\linewidth\linewidth\else\Gin@nat@width\fi}
\def\maxheight{\ifdim\Gin@nat@height>\textheight\textheight\else\Gin@nat@height\fi}
\makeatother
% Scale images if necessary, so that they will not overflow the page
% margins by default, and it is still possible to overwrite the defaults
% using explicit options in \includegraphics[width, height, ...]{}
\setkeys{Gin}{width=\maxwidth,height=\maxheight,keepaspectratio}
% Set default figure placement to htbp
\makeatletter
\def\fps@figure{htbp}
\makeatother
\setlength{\emergencystretch}{3em} % prevent overfull lines
\providecommand{\tightlist}{%
  \setlength{\itemsep}{0pt}\setlength{\parskip}{0pt}}
\setcounter{secnumdepth}{-\maxdimen} % remove section numbering
\ifLuaTeX
  \usepackage{selnolig}  % disable illegal ligatures
\fi
\IfFileExists{bookmark.sty}{\usepackage{bookmark}}{\usepackage{hyperref}}
\IfFileExists{xurl.sty}{\usepackage{xurl}}{} % add URL line breaks if available
\urlstyle{same}
\hypersetup{
  pdftitle={Trabalho Final Series Temporais},
  hidelinks,
  pdfcreator={LaTeX via pandoc}}

\title{Trabalho Final Series Temporais}
\author{}
\date{\vspace{-2.5em}}

\begin{document}
\maketitle

Aluno: Diogo Costa Hutner

Primeiro, carregando os modulos e os dados.

\begin{Shaded}
\begin{Highlighting}[]
\FunctionTok{library}\NormalTok{(readxl)}
\end{Highlighting}
\end{Shaded}

\begin{verbatim}
## Warning: package 'readxl' was built under R version 4.2.3
\end{verbatim}

\begin{Shaded}
\begin{Highlighting}[]
\FunctionTok{library}\NormalTok{(zoo)}
\end{Highlighting}
\end{Shaded}

\begin{verbatim}
## 
## Attaching package: 'zoo'
\end{verbatim}

\begin{verbatim}
## The following objects are masked from 'package:base':
## 
##     as.Date, as.Date.numeric
\end{verbatim}

\begin{Shaded}
\begin{Highlighting}[]
\FunctionTok{library}\NormalTok{(lubridate)}
\end{Highlighting}
\end{Shaded}

\begin{verbatim}
## 
## Attaching package: 'lubridate'
\end{verbatim}

\begin{verbatim}
## The following objects are masked from 'package:base':
## 
##     date, intersect, setdiff, union
\end{verbatim}

\begin{Shaded}
\begin{Highlighting}[]
\FunctionTok{library}\NormalTok{(ggplot2)}
\FunctionTok{library}\NormalTok{(tseries)}
\end{Highlighting}
\end{Shaded}

\begin{verbatim}
## Warning: package 'tseries' was built under R version 4.2.2
\end{verbatim}

\begin{verbatim}
## Registered S3 method overwritten by 'quantmod':
##   method            from
##   as.zoo.data.frame zoo
\end{verbatim}

\begin{Shaded}
\begin{Highlighting}[]
\FunctionTok{library}\NormalTok{(magick)}
\end{Highlighting}
\end{Shaded}

\begin{verbatim}
## Warning: package 'magick' was built under R version 4.2.3
\end{verbatim}

\begin{verbatim}
## Linking to ImageMagick 6.9.12.3
## Enabled features: cairo, freetype, fftw, ghostscript, heic, lcms, pango, raw, rsvg, webp
## Disabled features: fontconfig, x11
\end{verbatim}

\begin{Shaded}
\begin{Highlighting}[]
\FunctionTok{library}\NormalTok{(gridExtra)}
\end{Highlighting}
\end{Shaded}

\begin{verbatim}
## Warning: package 'gridExtra' was built under R version 4.2.3
\end{verbatim}

\begin{Shaded}
\begin{Highlighting}[]
\FunctionTok{library}\NormalTok{(ggfortify)}
\end{Highlighting}
\end{Shaded}

\begin{verbatim}
## Warning: package 'ggfortify' was built under R version 4.2.3
\end{verbatim}

\begin{Shaded}
\begin{Highlighting}[]
\FunctionTok{library}\NormalTok{(stats)}
\FunctionTok{library}\NormalTok{(forecast)}
\end{Highlighting}
\end{Shaded}

\begin{verbatim}
## Warning: package 'forecast' was built under R version 4.2.3
\end{verbatim}

\begin{verbatim}
## Registered S3 methods overwritten by 'forecast':
##   method                 from     
##   autoplot.Arima         ggfortify
##   autoplot.acf           ggfortify
##   autoplot.ar            ggfortify
##   autoplot.bats          ggfortify
##   autoplot.decomposed.ts ggfortify
##   autoplot.ets           ggfortify
##   autoplot.forecast      ggfortify
##   autoplot.stl           ggfortify
##   autoplot.ts            ggfortify
##   fitted.ar              ggfortify
##   fortify.ts             ggfortify
##   residuals.ar           ggfortify
\end{verbatim}

\begin{Shaded}
\begin{Highlighting}[]
\FunctionTok{library}\NormalTok{(vars)}
\end{Highlighting}
\end{Shaded}

\begin{verbatim}
## Warning: package 'vars' was built under R version 4.2.3
\end{verbatim}

\begin{verbatim}
## Carregando pacotes exigidos: MASS
\end{verbatim}

\begin{verbatim}
## Carregando pacotes exigidos: strucchange
\end{verbatim}

\begin{verbatim}
## Warning: package 'strucchange' was built under R version 4.2.2
\end{verbatim}

\begin{verbatim}
## Carregando pacotes exigidos: sandwich
\end{verbatim}

\begin{verbatim}
## Carregando pacotes exigidos: urca
\end{verbatim}

\begin{verbatim}
## Warning: package 'urca' was built under R version 4.2.3
\end{verbatim}

\begin{verbatim}
## Carregando pacotes exigidos: lmtest
\end{verbatim}

\begin{Shaded}
\begin{Highlighting}[]
\FunctionTok{library}\NormalTok{(lmtest)}
\FunctionTok{library}\NormalTok{(stargazer)}
\end{Highlighting}
\end{Shaded}

\begin{verbatim}
## 
## Please cite as:
\end{verbatim}

\begin{verbatim}
##  Hlavac, Marek (2022). stargazer: Well-Formatted Regression and Summary Statistics Tables.
\end{verbatim}

\begin{verbatim}
##  R package version 5.2.3. https://CRAN.R-project.org/package=stargazer
\end{verbatim}

\begin{Shaded}
\begin{Highlighting}[]
\FunctionTok{library}\NormalTok{(urca)}


\NormalTok{dados\_cointegracao }\OtherTok{\textless{}{-}} \FunctionTok{read\_excel}\NormalTok{(}\StringTok{"dados sobre cointegracao.xlsx"}\NormalTok{, }\AttributeTok{sheet =} \DecValTok{1}\NormalTok{)}
\NormalTok{dados\_cointegracao}\SpecialCharTok{$}\NormalTok{trim }\OtherTok{\textless{}{-}} \FunctionTok{as.yearqtr}\NormalTok{(dados\_cointegracao}\SpecialCharTok{$}\NormalTok{trim, }\AttributeTok{format =} \StringTok{"\%Yq\%q"}\NormalTok{)}
\NormalTok{dados\_cointegracao}
\end{Highlighting}
\end{Shaded}

\begin{verbatim}
## # A tibble: 136 x 3
##    trim          c     y
##    <yearqtr> <dbl> <dbl>
##  1 1947 Q1    4596  4886
##  2 1947 Q2    4655  4766
##  3 1947 Q3    4637  4855
##  4 1947 Q4    4609  4774
##  5 1948 Q1    4627  4869
##  6 1948 Q2    4658  4993
##  7 1948 Q3    4646  5070
##  8 1948 Q4    4660  5059
##  9 1949 Q1    4644  4938
## 10 1949 Q2    4681  4916
## # ... with 126 more rows
\end{verbatim}

\begin{Shaded}
\begin{Highlighting}[]
\NormalTok{dados\_sarima }\OtherTok{\textless{}{-}} \FunctionTok{read\_excel}\NormalTok{(}\StringTok{"Dados para replicar SARIMA  Vendas de Automoveis USA.xlsx"}\NormalTok{, }\AttributeTok{sheet =} \DecValTok{1}\NormalTok{)}
\FunctionTok{colnames}\NormalTok{(dados\_sarima) }\OtherTok{\textless{}{-}} \FunctionTok{c}\NormalTok{(}\StringTok{"trim"}\NormalTok{,}\StringTok{"RCAR6T"}\NormalTok{)}
\NormalTok{dados\_sarima}\SpecialCharTok{$}\NormalTok{trim }\OtherTok{\textless{}{-}} \FunctionTok{as.Date}\NormalTok{(dados\_sarima}\SpecialCharTok{$}\NormalTok{trim)}
\NormalTok{dados\_sarima}
\end{Highlighting}
\end{Shaded}

\begin{verbatim}
## # A tibble: 194 x 2
##    trim       RCAR6T
##    <date>      <dbl>
##  1 1980-01-01   806.
##  2 1980-02-01   812.
##  3 1980-03-01   895.
##  4 1980-04-01   743.
##  5 1980-05-01   697.
##  6 1980-06-01   702.
##  7 1980-07-01   773.
##  8 1980-08-01   686.
##  9 1980-09-01   674.
## 10 1980-10-01   848.
## # ... with 184 more rows
\end{verbatim}

\begin{Shaded}
\begin{Highlighting}[]
\NormalTok{dados\_var }\OtherTok{\textless{}{-}} \FunctionTok{read\_excel}\NormalTok{(}\StringTok{"Dados exemplo Modelo VAR sobre vendas veiculos e juros.xlsx"}\NormalTok{, }\AttributeTok{sheet=}\DecValTok{1}\NormalTok{)}
\end{Highlighting}
\end{Shaded}

\begin{verbatim}
## New names:
## * `` -> `...1`
\end{verbatim}

\begin{Shaded}
\begin{Highlighting}[]
\NormalTok{dados\_var}\SpecialCharTok{$}\NormalTok{txj }\OtherTok{\textless{}{-}} \FunctionTok{as.numeric}\NormalTok{(dados\_var}\SpecialCharTok{$}\NormalTok{txj)}
\FunctionTok{colnames}\NormalTok{(dados\_var) }\OtherTok{\textless{}{-}} \FunctionTok{c}\NormalTok{(}\StringTok{"trim"}\NormalTok{,}\StringTok{"vv"}\NormalTok{,}\StringTok{"txj"}\NormalTok{)}
\NormalTok{dados\_var}\SpecialCharTok{$}\NormalTok{trim }\OtherTok{\textless{}{-}} \FunctionTok{ymd}\NormalTok{(}\FunctionTok{paste0}\NormalTok{(dados\_var}\SpecialCharTok{$}\NormalTok{trim, }\StringTok{"01"}\NormalTok{))}
\NormalTok{cor\_linha }\OtherTok{\textless{}{-}} \StringTok{"\#11CC22"}
\end{Highlighting}
\end{Shaded}

\hypertarget{modelo-sarima}{%
\section{Modelo SARIMA}\label{modelo-sarima}}

De começo, ver os graficos como eles estão, depois plotar ele e fazer
teste de estacionaridade, apos isso ACF e PACF. Por fim postar um
summary do modelo

\begin{Shaded}
\begin{Highlighting}[]
\NormalTok{dados\_sarima}
\end{Highlighting}
\end{Shaded}

\begin{verbatim}
## # A tibble: 194 x 2
##    trim       RCAR6T
##    <date>      <dbl>
##  1 1980-01-01   806.
##  2 1980-02-01   812.
##  3 1980-03-01   895.
##  4 1980-04-01   743.
##  5 1980-05-01   697.
##  6 1980-06-01   702.
##  7 1980-07-01   773.
##  8 1980-08-01   686.
##  9 1980-09-01   674.
## 10 1980-10-01   848.
## # ... with 184 more rows
\end{verbatim}

\begin{Shaded}
\begin{Highlighting}[]
\NormalTok{grafico\_sar\_inicial }\OtherTok{\textless{}{-}} \FunctionTok{ggplot}\NormalTok{(dados\_sarima, }\FunctionTok{aes}\NormalTok{(}\AttributeTok{x =}\NormalTok{ trim, }\AttributeTok{y =}\NormalTok{ RCAR6T)) }\SpecialCharTok{+}
  \FunctionTok{geom\_line}\NormalTok{(}\AttributeTok{color =}\NormalTok{ cor\_linha) }\SpecialCharTok{+}
  \FunctionTok{labs}\NormalTok{(}\AttributeTok{x =} \StringTok{"Ano"}\NormalTok{, }\AttributeTok{y =} \StringTok{"Automoveis Vendidos"}\NormalTok{, }\AttributeTok{title =} \StringTok{"Gráfico p/ SARIMA Automoveis"}\NormalTok{)}
\NormalTok{grafico\_sar\_inicial}
\end{Highlighting}
\end{Shaded}

\includegraphics{Trabalho_files/figure-latex/unnamed-chunk-3-1.pdf}

\begin{Shaded}
\begin{Highlighting}[]
\NormalTok{results\_adf\_sar }\OtherTok{\textless{}{-}} \FunctionTok{adf.test}\NormalTok{(dados\_sarima}\SpecialCharTok{$}\NormalTok{RCAR6T)}
\NormalTok{results\_adf\_sar}
\end{Highlighting}
\end{Shaded}

\begin{verbatim}
## 
##  Augmented Dickey-Fuller Test
## 
## data:  dados_sarima$RCAR6T
## Dickey-Fuller = -3.5165, Lag order = 5, p-value = 0.04263
## alternative hypothesis: stationary
\end{verbatim}

\begin{Shaded}
\begin{Highlighting}[]
\NormalTok{acf\_result }\OtherTok{\textless{}{-}} \FunctionTok{acf}\NormalTok{(dados\_sarima}\SpecialCharTok{$}\NormalTok{RCAR6T, }\AttributeTok{lag=}\DecValTok{50}\NormalTok{)}
\end{Highlighting}
\end{Shaded}

\includegraphics{Trabalho_files/figure-latex/unnamed-chunk-5-1.pdf}

\begin{Shaded}
\begin{Highlighting}[]
\FunctionTok{plot}\NormalTok{(acf\_result, }\AttributeTok{main =} \StringTok{"Autocorrelation Function (ACF)"}\NormalTok{, }\AttributeTok{col =}\NormalTok{ cor\_linha)}
\end{Highlighting}
\end{Shaded}

\includegraphics{Trabalho_files/figure-latex/unnamed-chunk-5-2.pdf}

\begin{Shaded}
\begin{Highlighting}[]
\FunctionTok{as.data.frame}\NormalTok{(acf\_result}\SpecialCharTok{$}\NormalTok{acf)}
\end{Highlighting}
\end{Shaded}

\begin{verbatim}
##              V1
## 1   1.000000000
## 2   0.692407662
## 3   0.545943491
## 4   0.474553676
## 5   0.344853406
## 6   0.296686722
## 7   0.259254061
## 8   0.267136345
## 9   0.295922854
## 10  0.388487315
## 11  0.449843634
## 12  0.562022005
## 13  0.683810971
## 14  0.532090652
## 15  0.456969055
## 16  0.369957821
## 17  0.201843145
## 18  0.145811867
## 19  0.129349932
## 20  0.116806471
## 21  0.121860361
## 22  0.175400580
## 23  0.248984854
## 24  0.359575358
## 25  0.418024046
## 26  0.326650585
## 27  0.266767454
## 28  0.145587966
## 29  0.011514301
## 30 -0.032106115
## 31 -0.068127223
## 32 -0.071162973
## 33 -0.043899213
## 34  0.026445718
## 35  0.099066765
## 36  0.176426863
## 37  0.197113558
## 38  0.127505095
## 39  0.079658415
## 40 -0.015878380
## 41 -0.138207285
## 42 -0.216960267
## 43 -0.238303602
## 44 -0.229583544
## 45 -0.216144092
## 46 -0.145285920
## 47 -0.054220323
## 48  0.004082515
## 49  0.109588436
## 50  0.039685693
## 51 -0.054746174
\end{verbatim}

\begin{Shaded}
\begin{Highlighting}[]
\NormalTok{acf\_result }\OtherTok{\textless{}{-}} \FunctionTok{pacf}\NormalTok{(dados\_sarima}\SpecialCharTok{$}\NormalTok{RCAR6T, }\AttributeTok{lag=}\DecValTok{50}\NormalTok{, }\AttributeTok{col =}\NormalTok{ cor\_linha)}
\end{Highlighting}
\end{Shaded}

\includegraphics{Trabalho_files/figure-latex/unnamed-chunk-7-1.pdf}

\begin{Shaded}
\begin{Highlighting}[]
\FunctionTok{pacf}\NormalTok{(dados\_sarima}\SpecialCharTok{$}\NormalTok{RCAR6T, }\AttributeTok{lag=}\DecValTok{50}\NormalTok{, }\AttributeTok{col =}\NormalTok{ cor\_linha)}
\end{Highlighting}
\end{Shaded}

\includegraphics{Trabalho_files/figure-latex/unnamed-chunk-8-1.pdf}

\begin{Shaded}
\begin{Highlighting}[]
\NormalTok{modelo\_sarima }\OtherTok{\textless{}{-}} \FunctionTok{Arima}\NormalTok{(dados\_sarima}\SpecialCharTok{$}\NormalTok{RCAR6T, }\AttributeTok{order =} \FunctionTok{c}\NormalTok{(}\DecValTok{1}\NormalTok{, }\DecValTok{0}\NormalTok{, }\DecValTok{0}\NormalTok{), }\AttributeTok{seasonal =} \FunctionTok{list}\NormalTok{(}\AttributeTok{order =} \FunctionTok{c}\NormalTok{(}\DecValTok{1}\NormalTok{, }\DecValTok{0}\NormalTok{, }\DecValTok{0}\NormalTok{), }\AttributeTok{period =} \DecValTok{12}\NormalTok{))}

\FunctionTok{summary}\NormalTok{(modelo\_sarima)}
\end{Highlighting}
\end{Shaded}

\begin{verbatim}
## Series: dados_sarima$RCAR6T 
## ARIMA(1,0,0)(1,0,0)[12] with non-zero mean 
## 
## Coefficients:
##          ar1    sar1      mean
##       0.5438  0.5538  773.7203
## s.e.  0.0638  0.0626   24.2330
## 
## sigma^2 = 5580:  log likelihood = -1112.95
## AIC=2233.9   AICc=2234.11   BIC=2246.97
## 
## Training set error measures:
##                     ME     RMSE      MAE        MPE     MAPE      MASE
## Training set 0.1506749 74.12011 57.38731 -0.9516228 7.434101 0.7504201
##                     ACF1
## Training set -0.06595613
\end{verbatim}

\hypertarget{modelo-var}{%
\section{modelo VAR}\label{modelo-var}}

Primeiro, ver os dados, segundo os graficos, depois teste de
estacionaridade

\begin{Shaded}
\begin{Highlighting}[]
\NormalTok{dados\_var}
\end{Highlighting}
\end{Shaded}

\begin{verbatim}
## # A tibble: 144 x 3
##    trim           vv   txj
##    <date>      <dbl> <dbl>
##  1 1999-01-01  60069  2.18
##  2 1999-02-01  29474  2.38
##  3 1999-03-01 103085  3.33
##  4 1999-04-01  84816  2.35
##  5 1999-05-01  80241  2.02
##  6 1999-06-01  76260  1.67
##  7 1999-07-01  82960  1.66
##  8 1999-08-01  91896  1.57
##  9 1999-09-01  94546  1.49
## 10 1999-10-01  68898  1.38
## # ... with 134 more rows
\end{verbatim}

\begin{Shaded}
\begin{Highlighting}[]
\NormalTok{grafico\_var\_inicial\_vv }\OtherTok{\textless{}{-}} \FunctionTok{ggplot}\NormalTok{(dados\_var, }\FunctionTok{aes}\NormalTok{(}\AttributeTok{x =}\NormalTok{ trim, }\AttributeTok{y =}\NormalTok{ vv)) }\SpecialCharTok{+}
  \FunctionTok{geom\_line}\NormalTok{(}\AttributeTok{color =}\NormalTok{ cor\_linha) }\SpecialCharTok{+}
  \FunctionTok{labs}\NormalTok{(}\AttributeTok{x =} \StringTok{"Ano"}\NormalTok{, }\AttributeTok{y =} \StringTok{"Automoveis Vendidos"}\NormalTok{, }\AttributeTok{title =} \StringTok{"Venda de Veiculos"}\NormalTok{)}
\NormalTok{grafico\_var\_inicial\_vv}
\end{Highlighting}
\end{Shaded}

\includegraphics{Trabalho_files/figure-latex/unnamed-chunk-11-1.pdf}

\begin{Shaded}
\begin{Highlighting}[]
\NormalTok{grafico\_var\_inicial\_vv }\OtherTok{\textless{}{-}} \FunctionTok{ggplot}\NormalTok{(dados\_var, }\FunctionTok{aes}\NormalTok{(}\AttributeTok{x =}\NormalTok{ trim, }\AttributeTok{y =}\NormalTok{ txj)) }\SpecialCharTok{+}
  \FunctionTok{geom\_line}\NormalTok{(}\AttributeTok{color =}\NormalTok{ cor\_linha) }\SpecialCharTok{+}
  \FunctionTok{labs}\NormalTok{(}\AttributeTok{x =} \StringTok{"Ano"}\NormalTok{, }\AttributeTok{y =} \StringTok{"Taxa de Juros"}\NormalTok{, }\AttributeTok{title =} \StringTok{"Taxa de Juros"}\NormalTok{)}

\NormalTok{grafico\_var\_inicial\_vv}
\end{Highlighting}
\end{Shaded}

\includegraphics{Trabalho_files/figure-latex/unnamed-chunk-12-1.pdf}
Teste de Dickey-Fuller sobre vendas de carros e apos sobre taxa de juros

\begin{Shaded}
\begin{Highlighting}[]
\FunctionTok{adf.test}\NormalTok{(dados\_var}\SpecialCharTok{$}\NormalTok{vv)}
\end{Highlighting}
\end{Shaded}

\begin{verbatim}
## 
##  Augmented Dickey-Fuller Test
## 
## data:  dados_var$vv
## Dickey-Fuller = -3.7548, Lag order = 5, p-value = 0.02323
## alternative hypothesis: stationary
\end{verbatim}

\begin{Shaded}
\begin{Highlighting}[]
\FunctionTok{adf.test}\NormalTok{(dados\_var}\SpecialCharTok{$}\NormalTok{txj)}
\end{Highlighting}
\end{Shaded}

\begin{verbatim}
## 
##  Augmented Dickey-Fuller Test
## 
## data:  dados_var$txj
## Dickey-Fuller = -3.032, Lag order = 5, p-value = 0.1467
## alternative hypothesis: stationary
\end{verbatim}

Montando um modelo VAR:

\begin{Shaded}
\begin{Highlighting}[]
\NormalTok{modelo\_var }\OtherTok{\textless{}{-}} \FunctionTok{VAR}\NormalTok{(dados\_var[,}\FunctionTok{c}\NormalTok{(}\StringTok{"txj"}\NormalTok{,}\StringTok{\textquotesingle{}vv\textquotesingle{}}\NormalTok{)], }\AttributeTok{p =} \DecValTok{1}\NormalTok{)}
\FunctionTok{summary}\NormalTok{(modelo\_var)}
\end{Highlighting}
\end{Shaded}

\begin{verbatim}
## 
## VAR Estimation Results:
## ========================= 
## Endogenous variables: txj, vv 
## Deterministic variables: const 
## Sample size: 143 
## Log Likelihood: -1526.109 
## Roots of the characteristic polynomial:
## 0.9594 0.499
## Call:
## VAR(y = dados_var[, c("txj", "vv")], p = 1)
## 
## 
## Estimation results for equation txj: 
## ==================================== 
## txj = txj.l1 + vv.l1 + const 
## 
##          Estimate Std. Error t value Pr(>|t|)    
## txj.l1  7.095e-01  4.964e-02  14.293  < 2e-16 ***
## vv.l1  -2.294e-06  4.792e-07  -4.788 4.23e-06 ***
## const   6.425e-01  1.140e-01   5.635 9.29e-08 ***
## ---
## Signif. codes:  0 '***' 0.001 '**' 0.01 '*' 0.05 '.' 0.1 ' ' 1
## 
## 
## Residual standard error: 0.1634 on 140 degrees of freedom
## Multiple R-Squared: 0.8297,  Adjusted R-squared: 0.8273 
## F-statistic: 341.1 on 2 and 140 DF,  p-value: < 2.2e-16 
## 
## 
## Estimation results for equation vv: 
## =================================== 
## vv = txj.l1 + vv.l1 + const 
## 
##          Estimate Std. Error t value Pr(>|t|)    
## txj.l1 -2.293e+04  5.023e+03  -4.565 1.09e-05 ***
## vv.l1   7.489e-01  4.849e-02  15.443  < 2e-16 ***
## const   6.095e+04  1.154e+04   5.283 4.76e-07 ***
## ---
## Signif. codes:  0 '***' 0.001 '**' 0.01 '*' 0.05 '.' 0.1 ' ' 1
## 
## 
## Residual standard error: 16540 on 140 degrees of freedom
## Multiple R-Squared: 0.8437,  Adjusted R-squared: 0.8415 
## F-statistic: 377.8 on 2 and 140 DF,  p-value: < 2.2e-16 
## 
## 
## 
## Covariance matrix of residuals:
##           txj        vv
## txj   0.02671 8.073e+02
## vv  807.28055 2.735e+08
## 
## Correlation matrix of residuals:
##        txj     vv
## txj 1.0000 0.2987
## vv  0.2987 1.0000
\end{verbatim}

Causalidade de Granger

\begin{Shaded}
\begin{Highlighting}[]
\FunctionTok{grangertest}\NormalTok{(txj }\SpecialCharTok{\textasciitilde{}}\NormalTok{ vv, }\AttributeTok{data =}\NormalTok{ dados\_var)}
\end{Highlighting}
\end{Shaded}

\begin{verbatim}
## Granger causality test
## 
## Model 1: txj ~ Lags(txj, 1:1) + Lags(vv, 1:1)
## Model 2: txj ~ Lags(txj, 1:1)
##   Res.Df Df      F    Pr(>F)    
## 1    140                        
## 2    141 -1 22.926 4.232e-06 ***
## ---
## Signif. codes:  0 '***' 0.001 '**' 0.01 '*' 0.05 '.' 0.1 ' ' 1
\end{verbatim}

\begin{Shaded}
\begin{Highlighting}[]
\FunctionTok{print}\NormalTok{(}\StringTok{\textquotesingle{}{-}{-}{-}{-}{-}{-}{-}{-}{-}{-}{-}{-}{-}\textquotesingle{}}\NormalTok{)}
\end{Highlighting}
\end{Shaded}

\begin{verbatim}
## [1] "-------------"
\end{verbatim}

\begin{Shaded}
\begin{Highlighting}[]
\FunctionTok{grangertest}\NormalTok{(vv }\SpecialCharTok{\textasciitilde{}}\NormalTok{ txj, }\AttributeTok{data =}\NormalTok{ dados\_var)}
\end{Highlighting}
\end{Shaded}

\begin{verbatim}
## Granger causality test
## 
## Model 1: vv ~ Lags(vv, 1:1) + Lags(txj, 1:1)
## Model 2: vv ~ Lags(vv, 1:1)
##   Res.Df Df      F    Pr(>F)    
## 1    140                        
## 2    141 -1 20.836 1.085e-05 ***
## ---
## Signif. codes:  0 '***' 0.001 '**' 0.01 '*' 0.05 '.' 0.1 ' ' 1
\end{verbatim}

Plotando graficos de IRF

\begin{Shaded}
\begin{Highlighting}[]
\NormalTok{irf\_txj }\OtherTok{\textless{}{-}} \FunctionTok{irf}\NormalTok{(modelo\_var,}\AttributeTok{impulse =} \StringTok{"txj"}\NormalTok{, }\AttributeTok{response=}\FunctionTok{c}\NormalTok{(}\StringTok{"txj"}\NormalTok{,}\StringTok{"vv"}\NormalTok{), }\AttributeTok{n.ahead =} \DecValTok{10}\NormalTok{)}

\NormalTok{dados\_irf\_txj }\OtherTok{\textless{}{-}} \FunctionTok{data.frame}\NormalTok{(}
  \AttributeTok{txj =}\NormalTok{ irf\_txj}\SpecialCharTok{$}\NormalTok{irf}\SpecialCharTok{$}\NormalTok{txj[, }\StringTok{\textquotesingle{}txj\textquotesingle{}}\NormalTok{],}
  \AttributeTok{vv =}\NormalTok{ irf\_txj}\SpecialCharTok{$}\NormalTok{irf}\SpecialCharTok{$}\NormalTok{txj[, }\StringTok{\textquotesingle{}vv\textquotesingle{}}\NormalTok{],}
  \AttributeTok{lower\_txj =}\NormalTok{ irf\_txj}\SpecialCharTok{$}\NormalTok{Lower}\SpecialCharTok{$}\NormalTok{txj[, }\StringTok{\textquotesingle{}txj\textquotesingle{}}\NormalTok{],}
  \AttributeTok{upper\_txj =}\NormalTok{ irf\_txj}\SpecialCharTok{$}\NormalTok{Upper}\SpecialCharTok{$}\NormalTok{txj[, }\StringTok{\textquotesingle{}txj\textquotesingle{}}\NormalTok{],}
  \AttributeTok{lower\_vv =}\NormalTok{ irf\_txj}\SpecialCharTok{$}\NormalTok{Lower}\SpecialCharTok{$}\NormalTok{txj[, }\StringTok{\textquotesingle{}vv\textquotesingle{}}\NormalTok{],}
  \AttributeTok{upper\_vv =}\NormalTok{ irf\_txj}\SpecialCharTok{$}\NormalTok{Upper}\SpecialCharTok{$}\NormalTok{txj[, }\StringTok{\textquotesingle{}vv\textquotesingle{}}\NormalTok{]}
\NormalTok{)}

\NormalTok{grafico\_irf\_txj\_txj }\OtherTok{\textless{}{-}} \FunctionTok{ggplot}\NormalTok{(dados\_irf\_txj, }\FunctionTok{aes}\NormalTok{(}\AttributeTok{x =} \FunctionTok{seq\_along}\NormalTok{(txj))) }\SpecialCharTok{+}
  \FunctionTok{geom\_line}\NormalTok{(}\FunctionTok{aes}\NormalTok{(}\AttributeTok{y =}\NormalTok{ txj, }\AttributeTok{color =} \StringTok{"Resposta ao Impulso"}\NormalTok{), }\AttributeTok{size =} \DecValTok{1}\NormalTok{) }\SpecialCharTok{+}
  \FunctionTok{geom\_line}\NormalTok{(}\FunctionTok{aes}\NormalTok{(}\AttributeTok{y =}\NormalTok{ lower\_txj, }\AttributeTok{color =} \StringTok{"Limite Inferior"}\NormalTok{), }\AttributeTok{linetype =} \StringTok{"dashed"}\NormalTok{, }\AttributeTok{color =} \StringTok{"red"}\NormalTok{) }\SpecialCharTok{+}
  \FunctionTok{geom\_line}\NormalTok{(}\FunctionTok{aes}\NormalTok{(}\AttributeTok{y =}\NormalTok{ upper\_txj, }\AttributeTok{color =} \StringTok{"Limite Superior"}\NormalTok{), }\AttributeTok{linetype =} \StringTok{"dashed"}\NormalTok{, }\AttributeTok{color =} \StringTok{"red"}\NormalTok{) }\SpecialCharTok{+}
  \FunctionTok{scale\_color\_manual}\NormalTok{(}\AttributeTok{values =} \FunctionTok{c}\NormalTok{(}\StringTok{"Impulse Response"} \OtherTok{=}\NormalTok{ cor\_linha, }\StringTok{"Lower Bound"} \OtherTok{=} \StringTok{"red"}\NormalTok{, }\StringTok{"Upper Bound"} \OtherTok{=} \StringTok{"red"}\NormalTok{)) }\SpecialCharTok{+}
  \FunctionTok{labs}\NormalTok{(}\AttributeTok{x =} \StringTok{"Lag"}\NormalTok{, }\AttributeTok{y =} \StringTok{"Taxa Juros"}\NormalTok{, }\AttributeTok{title =} \StringTok{"Impulse Response Function da taxa de juros na taixa de juros"}\NormalTok{) }\SpecialCharTok{+}
  \FunctionTok{theme\_minimal}\NormalTok{()}


\NormalTok{grafico\_irf\_txj\_vv }\OtherTok{\textless{}{-}} \FunctionTok{ggplot}\NormalTok{(dados\_irf\_txj, }\FunctionTok{aes}\NormalTok{(}\AttributeTok{x =} \FunctionTok{seq\_along}\NormalTok{(txj))) }\SpecialCharTok{+}
  \FunctionTok{geom\_line}\NormalTok{(}\FunctionTok{aes}\NormalTok{(}\AttributeTok{y =}\NormalTok{ vv, }\AttributeTok{color =} \StringTok{"Resposta ao Impulso"}\NormalTok{), }\AttributeTok{size =} \DecValTok{1}\NormalTok{) }\SpecialCharTok{+}
  \FunctionTok{geom\_line}\NormalTok{(}\FunctionTok{aes}\NormalTok{(}\AttributeTok{y =}\NormalTok{ lower\_vv, }\AttributeTok{color =} \StringTok{"Limite Inferior"}\NormalTok{), }\AttributeTok{linetype =} \StringTok{"dashed"}\NormalTok{, }\AttributeTok{color =} \StringTok{"red"}\NormalTok{) }\SpecialCharTok{+}
  \FunctionTok{geom\_line}\NormalTok{(}\FunctionTok{aes}\NormalTok{(}\AttributeTok{y =}\NormalTok{ upper\_vv, }\AttributeTok{color =} \StringTok{"Limite Superior"}\NormalTok{), }\AttributeTok{linetype =} \StringTok{"dashed"}\NormalTok{, }\AttributeTok{color =} \StringTok{"red"}\NormalTok{) }\SpecialCharTok{+}
  \FunctionTok{scale\_color\_manual}\NormalTok{(}\AttributeTok{values =} \FunctionTok{c}\NormalTok{(}\StringTok{"Impulse Response"} \OtherTok{=}\NormalTok{ cor\_linha, }\StringTok{"Lower Bound"} \OtherTok{=} \StringTok{"red"}\NormalTok{, }\StringTok{"Upper Bound"} \OtherTok{=} \StringTok{"red"}\NormalTok{)) }\SpecialCharTok{+}
  \FunctionTok{labs}\NormalTok{(}\AttributeTok{x =} \StringTok{"Lag"}\NormalTok{, }\AttributeTok{y =} \StringTok{"Venda de Carros"}\NormalTok{, }\AttributeTok{title =} \StringTok{"Impulse Response Function da taixa de juros na venda de carros"}\NormalTok{) }\SpecialCharTok{+}
  \FunctionTok{theme\_minimal}\NormalTok{()}
\FunctionTok{print}\NormalTok{(grafico\_irf\_txj\_txj)}
\end{Highlighting}
\end{Shaded}

\includegraphics{Trabalho_files/figure-latex/unnamed-chunk-17-1.pdf}

\begin{Shaded}
\begin{Highlighting}[]
\FunctionTok{print}\NormalTok{(grafico\_irf\_txj\_vv)}
\end{Highlighting}
\end{Shaded}

\includegraphics{Trabalho_files/figure-latex/unnamed-chunk-18-1.pdf}

\begin{Shaded}
\begin{Highlighting}[]
\NormalTok{irf\_vv }\OtherTok{\textless{}{-}} \FunctionTok{irf}\NormalTok{(modelo\_var,}\AttributeTok{impulse =} \StringTok{"vv"}\NormalTok{, }\AttributeTok{response=}\FunctionTok{c}\NormalTok{(}\StringTok{"txj"}\NormalTok{,}\StringTok{"vv"}\NormalTok{), }\AttributeTok{n.ahead =} \DecValTok{10}\NormalTok{)}

\NormalTok{dados\_irf\_vv }\OtherTok{\textless{}{-}} \FunctionTok{data.frame}\NormalTok{(}
  \AttributeTok{txj =}\NormalTok{ irf\_vv}\SpecialCharTok{$}\NormalTok{irf}\SpecialCharTok{$}\NormalTok{vv[, }\StringTok{\textquotesingle{}txj\textquotesingle{}}\NormalTok{],}
  \AttributeTok{vv =}\NormalTok{ irf\_vv}\SpecialCharTok{$}\NormalTok{irf}\SpecialCharTok{$}\NormalTok{vv[, }\StringTok{\textquotesingle{}vv\textquotesingle{}}\NormalTok{],}
  \AttributeTok{lower\_txj =}\NormalTok{ irf\_vv}\SpecialCharTok{$}\NormalTok{Lower}\SpecialCharTok{$}\NormalTok{vv[, }\StringTok{\textquotesingle{}txj\textquotesingle{}}\NormalTok{],}
  \AttributeTok{upper\_txj =}\NormalTok{ irf\_vv}\SpecialCharTok{$}\NormalTok{Upper}\SpecialCharTok{$}\NormalTok{vv[, }\StringTok{\textquotesingle{}txj\textquotesingle{}}\NormalTok{],}
  \AttributeTok{lower\_vv =}\NormalTok{ irf\_vv}\SpecialCharTok{$}\NormalTok{Lower}\SpecialCharTok{$}\NormalTok{vv[, }\StringTok{\textquotesingle{}vv\textquotesingle{}}\NormalTok{],}
  \AttributeTok{upper\_vv =}\NormalTok{ irf\_vv}\SpecialCharTok{$}\NormalTok{Upper}\SpecialCharTok{$}\NormalTok{vv[, }\StringTok{\textquotesingle{}vv\textquotesingle{}}\NormalTok{]}
\NormalTok{)}

\NormalTok{grafico\_irf\_vv\_txj }\OtherTok{\textless{}{-}} \FunctionTok{ggplot}\NormalTok{(dados\_irf\_vv, }\FunctionTok{aes}\NormalTok{(}\AttributeTok{x =} \FunctionTok{seq\_along}\NormalTok{(txj))) }\SpecialCharTok{+}
  \FunctionTok{geom\_line}\NormalTok{(}\FunctionTok{aes}\NormalTok{(}\AttributeTok{y =}\NormalTok{ txj, }\AttributeTok{color =} \StringTok{"Resposta ao Impulso"}\NormalTok{), }\AttributeTok{size =} \DecValTok{1}\NormalTok{) }\SpecialCharTok{+}
  \FunctionTok{geom\_line}\NormalTok{(}\FunctionTok{aes}\NormalTok{(}\AttributeTok{y =}\NormalTok{ lower\_txj, }\AttributeTok{color =} \StringTok{"Limite Inferior"}\NormalTok{), }\AttributeTok{linetype =} \StringTok{"dashed"}\NormalTok{, }\AttributeTok{color =} \StringTok{"red"}\NormalTok{) }\SpecialCharTok{+}
  \FunctionTok{geom\_line}\NormalTok{(}\FunctionTok{aes}\NormalTok{(}\AttributeTok{y =}\NormalTok{ upper\_txj, }\AttributeTok{color =} \StringTok{"Limite Superior"}\NormalTok{), }\AttributeTok{linetype =} \StringTok{"dashed"}\NormalTok{, }\AttributeTok{color =} \StringTok{"red"}\NormalTok{) }\SpecialCharTok{+}
  \FunctionTok{scale\_color\_manual}\NormalTok{(}\AttributeTok{values =} \FunctionTok{c}\NormalTok{(}\StringTok{"Impulse Response"} \OtherTok{=}\NormalTok{ cor\_linha, }\StringTok{"Lower Bound"} \OtherTok{=} \StringTok{"red"}\NormalTok{, }\StringTok{"Upper Bound"} \OtherTok{=} \StringTok{"red"}\NormalTok{)) }\SpecialCharTok{+}
  \FunctionTok{labs}\NormalTok{(}\AttributeTok{x =} \StringTok{"Lag"}\NormalTok{, }\AttributeTok{y =} \StringTok{"Taxa Juros"}\NormalTok{, }\AttributeTok{title =} \StringTok{"Impulse Response Function da venda de carro na taxa de juros"}\NormalTok{) }\SpecialCharTok{+}
  \FunctionTok{theme\_minimal}\NormalTok{()}


\NormalTok{grafico\_irf\_vv\_txj }\OtherTok{\textless{}{-}} \FunctionTok{ggplot}\NormalTok{(dados\_irf\_vv, }\FunctionTok{aes}\NormalTok{(}\AttributeTok{x =} \FunctionTok{seq\_along}\NormalTok{(txj))) }\SpecialCharTok{+}
  \FunctionTok{geom\_line}\NormalTok{(}\FunctionTok{aes}\NormalTok{(}\AttributeTok{y =}\NormalTok{ vv, }\AttributeTok{color =} \StringTok{"Resposta ao Impulso"}\NormalTok{), }\AttributeTok{size =} \DecValTok{1}\NormalTok{) }\SpecialCharTok{+}
  \FunctionTok{geom\_line}\NormalTok{(}\FunctionTok{aes}\NormalTok{(}\AttributeTok{y =}\NormalTok{ lower\_vv, }\AttributeTok{color =} \StringTok{"Limite Inferior"}\NormalTok{), }\AttributeTok{linetype =} \StringTok{"dashed"}\NormalTok{, }\AttributeTok{color =} \StringTok{"red"}\NormalTok{) }\SpecialCharTok{+}
  \FunctionTok{geom\_line}\NormalTok{(}\FunctionTok{aes}\NormalTok{(}\AttributeTok{y =}\NormalTok{ upper\_vv, }\AttributeTok{color =} \StringTok{"Limite Superior"}\NormalTok{), }\AttributeTok{linetype =} \StringTok{"dashed"}\NormalTok{, }\AttributeTok{color =} \StringTok{"red"}\NormalTok{) }\SpecialCharTok{+}
  \FunctionTok{scale\_color\_manual}\NormalTok{(}\AttributeTok{values =} \FunctionTok{c}\NormalTok{(}\StringTok{"Impulse Response"} \OtherTok{=}\NormalTok{ cor\_linha, }\StringTok{"Lower Bound"} \OtherTok{=} \StringTok{"red"}\NormalTok{, }\StringTok{"Upper Bound"} \OtherTok{=} \StringTok{"red"}\NormalTok{)) }\SpecialCharTok{+}
  \FunctionTok{labs}\NormalTok{(}\AttributeTok{x =} \StringTok{"Lag"}\NormalTok{, }\AttributeTok{y =} \StringTok{"Venda de Carros"}\NormalTok{, }\AttributeTok{title =} \StringTok{"Impulse Response Function da venda de carros na venda de carros"}\NormalTok{) }\SpecialCharTok{+}
  \FunctionTok{theme\_minimal}\NormalTok{()}
\NormalTok{grafico\_irf\_vv\_txj}
\end{Highlighting}
\end{Shaded}

\includegraphics{Trabalho_files/figure-latex/unnamed-chunk-19-1.pdf}

\begin{Shaded}
\begin{Highlighting}[]
\NormalTok{grafico\_irf\_vv\_txj}
\end{Highlighting}
\end{Shaded}

\includegraphics{Trabalho_files/figure-latex/unnamed-chunk-20-1.pdf} \#
Cointegração e VEC

Por questão de espaço, não irei realizar visivelmente aq o
Dickey-Fuller, mas mostrou que não é estacionaria, então vamos
diferenciar

\begin{Shaded}
\begin{Highlighting}[]
\NormalTok{dados\_dif\_cointegracao }\OtherTok{\textless{}{-}} \FunctionTok{cbind}\NormalTok{(}\FunctionTok{diff}\NormalTok{(dados\_cointegracao}\SpecialCharTok{$}\NormalTok{c), }\FunctionTok{diff}\NormalTok{(dados\_cointegracao}\SpecialCharTok{$}\NormalTok{y))}
\FunctionTok{colnames}\NormalTok{(dados\_dif\_cointegracao) }\OtherTok{\textless{}{-}} \FunctionTok{c}\NormalTok{(}\StringTok{"c"}\NormalTok{,}\StringTok{"y"}\NormalTok{)}
\FunctionTok{rownames}\NormalTok{(dados\_dif\_cointegracao) }\OtherTok{\textless{}{-}}\NormalTok{ dados\_cointegracao}\SpecialCharTok{$}\NormalTok{trim[}\SpecialCharTok{{-}}\DecValTok{1}\NormalTok{]}
\NormalTok{df\_dados\_dif\_cointegracao }\OtherTok{\textless{}{-}} \FunctionTok{fortify}\NormalTok{(dados\_dif\_cointegracao)}
\NormalTok{df\_dados\_dif\_cointegracao}
\end{Highlighting}
\end{Shaded}

\begin{verbatim}
##            c    y
## 1947.25   59 -120
## 1947.5   -18   89
## 1947.75  -28  -81
## 1948      18   95
## 1948.25   31  124
## 1948.5   -12   77
## 1948.75   14  -11
## 1949     -16 -121
## 1949.25   37  -22
## 1949.5   -30  -13
## 1949.75   14   -6
## 1950      53  331
## 1950.25   76  -45
## 1950.5   172    1
## 1950.75 -117   91
## 1951      72  -60
## 1951.25 -111  118
## 1951.5    24   14
## 1951.75    7  -15
## 1952      -3  -27
## 1952.25   67   30
## 1952.5     7   92
## 1952.75   88   14
## 1953      43   45
## 1953.25    5   60
## 1953.5   -29  -36
## 1953.75  -21   -3
## 1954       3  -18
## 1954.25   26  -49
## 1954.5    55   62
## 1954.75   67   79
## 1955      57    9
## 1955.25   64   89
## 1955.5    29   72
## 1955.75   63   76
## 1956      -8   24
## 1956.25  -15   20
## 1956.5    -9    3
## 1956.75   29   43
## 1957      13  -15
## 1957.25  -11   16
## 1957.5    13    5
## 1957.75   -8  -35
## 1958     -72  -61
## 1958.25   34   21
## 1958.5    57  100
## 1958.75   31   52
## 1959      63   -9
## 1959.25   51   75
## 1959.5    27  -56
## 1959.75   -7   24
## 1960      -1   17
## 1960.25   52   13
## 1960.5   -48  -28
## 1960.75  -13  -42
## 1961     -12   34
## 1961.25   56   69
## 1961.5   -19   27
## 1961.75   76   79
## 1962      36   40
## 1962.25   35   29
## 1962.5    28   13
## 1962.75   45    1
## 1963      25   38
## 1963.25   14   19
## 1963.5    59   42
## 1963.75   18   73
## 1964      95  108
## 1964.25   81  156
## 1964.5    84   61
## 1964.75   -2   51
## 1965      85   25
## 1965.25   54   83
## 1965.5    86  164
## 1965.75  135   93
## 1966      57   25
## 1966.25   11   23
## 1966.5    43   54
## 1966.75    1   47
## 1967      33   98
## 1967.25   69   52
## 1967.5    17   40
## 1967.75   21   32
## 1968     120   83
## 1968.25   75  103
## 1968.5   106  -19
## 1968.75   10   29
## 1969      61  -10
## 1969.25   29   73
## 1969.5    20  140
## 1969.75   38   40
## 1970      29   17
## 1970.25   11  113
## 1970.5    38   69
## 1970.75  -44  -48
## 1971      90  101
## 1971.25   38   91
## 1971.5    17  -15
## 1971.75   70    1
## 1972      86   34
## 1972.25  106   60
## 1972.5    84  139
## 1972.75  161  300
## 1973      86   94
## 1973.25  -32   48
## 1973.5    16   46
## 1973.75  -56   71
## 1974    -105 -182
## 1974.25   27 -108
## 1974.5    26   26
## 1974.75 -124  -52
## 1975      43 -107
## 1975.25  112  408
## 1975.5    61 -168
## 1975.75   58   60
## 1976     151  118
## 1976.25   41   26
## 1976.5    71   36
## 1976.75  104   50
## 1977     100    0
## 1977.25    0   81
## 1977.5    63  143
## 1977.75   96   45
## 1978      19   93
## 1978.25  148  131
## 1978.5    18   33
## 1978.75   60   86
## 1979      10   40
## 1979.25  -35  -70
## 1979.5    33    1
## 1979.75   18  -30
## 1980     -35   26
## 1980.25 -209 -205
## 1980.5    71   65
## 1980.75   57  110
\end{verbatim}

\begin{Shaded}
\begin{Highlighting}[]
\NormalTok{grafico\_var\_inicial\_vv }\OtherTok{\textless{}{-}} \FunctionTok{ggplot}\NormalTok{(df\_dados\_dif\_cointegracao, }\FunctionTok{aes}\NormalTok{(}\AttributeTok{x =} \FunctionTok{index}\NormalTok{(dados\_dif\_cointegracao), }\AttributeTok{y =}\NormalTok{ c)) }\SpecialCharTok{+}
  \FunctionTok{geom\_line}\NormalTok{(}\AttributeTok{color =}\NormalTok{ cor\_linha) }\SpecialCharTok{+}
  \FunctionTok{labs}\NormalTok{(}\AttributeTok{x =} \StringTok{"Ano"}\NormalTok{, }\AttributeTok{y =} \StringTok{"Diff Consumo Norte Americano"}\NormalTok{, }\AttributeTok{title =} \StringTok{"Consumo Norte Americano"}\NormalTok{)}

\NormalTok{grafico\_var\_inicial\_vv}
\end{Highlighting}
\end{Shaded}

\includegraphics{Trabalho_files/figure-latex/unnamed-chunk-22-1.pdf}

\begin{Shaded}
\begin{Highlighting}[]
\NormalTok{grafico\_var\_inicial\_vv }\OtherTok{\textless{}{-}} \FunctionTok{ggplot}\NormalTok{(df\_dados\_dif\_cointegracao, }\FunctionTok{aes}\NormalTok{(}\AttributeTok{x =} \FunctionTok{index}\NormalTok{(dados\_dif\_cointegracao), }\AttributeTok{y =}\NormalTok{ y)) }\SpecialCharTok{+}
  \FunctionTok{geom\_line}\NormalTok{(}\AttributeTok{color =}\NormalTok{ cor\_linha) }\SpecialCharTok{+}
  \FunctionTok{labs}\NormalTok{(}\AttributeTok{x =} \StringTok{"Ano"}\NormalTok{, }\AttributeTok{y =} \StringTok{"Diff PIB Norte Americano"}\NormalTok{, }\AttributeTok{title =} \StringTok{"PIB Norte Americano"}\NormalTok{)}
\NormalTok{grafico\_var\_inicial\_vv}
\end{Highlighting}
\end{Shaded}

\includegraphics{Trabalho_files/figure-latex/unnamed-chunk-23-1.pdf}
Sumario/Resumo do modelo de cointegração:

\begin{Shaded}
\begin{Highlighting}[]
\NormalTok{modelo\_cointegracao }\OtherTok{\textless{}{-}} \FunctionTok{lm}\NormalTok{(y }\SpecialCharTok{\textasciitilde{}}\NormalTok{ c, }\AttributeTok{data =} \FunctionTok{as.data.frame}\NormalTok{(dados\_dif\_cointegracao))}
\FunctionTok{summary}\NormalTok{(modelo\_cointegracao)}
\end{Highlighting}
\end{Shaded}

\begin{verbatim}
## 
## Call:
## lm(formula = y ~ c, data = as.data.frame(dados_dif_cointegracao))
## 
## Residuals:
##     Min      1Q  Median      3Q     Max 
## -223.14  -36.44    1.08   31.42  320.60 
## 
## Coefficients:
##             Estimate Std. Error t value Pr(>|t|)    
## (Intercept)  16.5551     7.1995   2.299    0.023 *  
## c             0.6326     0.1134   5.579  1.3e-07 ***
## ---
## Signif. codes:  0 '***' 0.001 '**' 0.01 '*' 0.05 '.' 0.1 ' ' 1
## 
## Residual standard error: 72.85 on 133 degrees of freedom
## Multiple R-squared:  0.1897, Adjusted R-squared:  0.1836 
## F-statistic: 31.13 on 1 and 133 DF,  p-value: 1.3e-07
\end{verbatim}

Analisando os residuos

\begin{Shaded}
\begin{Highlighting}[]
\NormalTok{residuos }\OtherTok{\textless{}{-}} \FunctionTok{fortify}\NormalTok{(}\FunctionTok{as.data.frame}\NormalTok{(}\FunctionTok{residuals}\NormalTok{(modelo\_cointegracao)))}
\FunctionTok{colnames}\NormalTok{(residuos) }\OtherTok{\textless{}{-}} \FunctionTok{c}\NormalTok{(}\StringTok{\textquotesingle{}res\textquotesingle{}}\NormalTok{)}
\FunctionTok{rownames}\NormalTok{(residuos) }\OtherTok{\textless{}{-}}\NormalTok{ dados\_cointegracao}\SpecialCharTok{$}\NormalTok{trim[}\SpecialCharTok{{-}}\DecValTok{1}\NormalTok{]}

\NormalTok{grafico\_resi }\OtherTok{\textless{}{-}} \FunctionTok{ggplot}\NormalTok{(residuos, }\FunctionTok{aes}\NormalTok{(}\AttributeTok{x =} \FunctionTok{rownames}\NormalTok{(residuos), }\AttributeTok{y =}\NormalTok{ res)) }\SpecialCharTok{+}
  \FunctionTok{geom\_point}\NormalTok{(}\AttributeTok{color =}\NormalTok{ cor\_linha) }\SpecialCharTok{+}
  \FunctionTok{labs}\NormalTok{(}\AttributeTok{x =} \StringTok{"Ano"}\NormalTok{, }\AttributeTok{y =} \StringTok{"Resíduo"}\NormalTok{, }\AttributeTok{title =} \StringTok{"Resíduos"}\NormalTok{)}
\NormalTok{grafico\_resi}
\end{Highlighting}
\end{Shaded}

\includegraphics{Trabalho_files/figure-latex/unnamed-chunk-25-1.pdf} ADF
sobre os residuos

\begin{Shaded}
\begin{Highlighting}[]
\FunctionTok{adf.test}\NormalTok{(residuos}\SpecialCharTok{$}\NormalTok{res)}
\end{Highlighting}
\end{Shaded}

\begin{verbatim}
## Warning in adf.test(residuos$res): p-value smaller than printed p-value
\end{verbatim}

\begin{verbatim}
## 
##  Augmented Dickey-Fuller Test
## 
## data:  residuos$res
## Dickey-Fuller = -5.7417, Lag order = 5, p-value = 0.01
## alternative hypothesis: stationary
\end{verbatim}

Por fim, um modelo VEC:

\begin{Shaded}
\begin{Highlighting}[]
\NormalTok{vec\_results }\OtherTok{\textless{}{-}} \FunctionTok{ca.jo}\NormalTok{(df\_dados\_dif\_cointegracao, }\AttributeTok{type =} \StringTok{"eigen"}\NormalTok{, }\AttributeTok{K=}\DecValTok{2}\NormalTok{)}
\FunctionTok{summary}\NormalTok{(vec\_results)}
\end{Highlighting}
\end{Shaded}

\begin{verbatim}
## 
## ###################### 
## # Johansen-Procedure # 
## ###################### 
## 
## Test type: maximal eigenvalue statistic (lambda max) , with linear trend 
## 
## Eigenvalues (lambda):
## [1] 0.4249105 0.2331722
## 
## Values of teststatistic and critical values of test:
## 
##           test 10pct  5pct  1pct
## r <= 1 | 35.31  6.50  8.18 11.65
## r = 0  | 73.58 12.91 14.90 19.19
## 
## Eigenvectors, normalised to first column:
## (These are the cointegration relations)
## 
##           c.l2       y.l2
## c.l2  1.000000 1.00000000
## y.l2 -1.140634 0.01272331
## 
## Weights W:
## (This is the loading matrix)
## 
##           c.l2       y.l2
## c.d -0.1053298 -0.6953155
## y.d  1.0828513 -0.5656351
\end{verbatim}

\end{document}
